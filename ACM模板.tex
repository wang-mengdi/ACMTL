\documentclass[UTF8,a4paper]{ctexart}
\usepackage{xcolor}
\usepackage{fancyhdr}
\usepackage{amsfonts}
%\usepackage{graphicx}
\usepackage{listings}
\usepackage{lastpage}
%\usepackage{multirow}
%\usepackage{makecell}
\usepackage{setspace}
\usepackage{indentfirst}

\usepackage{geometry}
\geometry{left=2.5cm,right=2.5cm,top=2.5cm,bottom=2.5cm}

\lstset{
  language=[ANSI]c,
  basicstyle=\small,
  numbers=left,
  keywordstyle=\color{blue},
  numberstyle={\tiny\color{lightgray}},
  stepnumber=1, %行号会逐行往上递增
  numbersep=5pt,
  showspaces=false,
  showtabs=false,
  tabsize=4,
  breaklines=tr,
  extendedchars=false %这一条命令可以解决代码跨页时,章节标题,页眉等汉字不显示的问题
}

\author{PKU cstdio,hzwer,Kuribohg}
\title{\Huge ACM模板}
\begin{document} 
\maketitle
\tableofcontents
\section{数学}

\subsection{线性筛}
\begin{lstlisting}[language=C]
void Linear_Sieve_Method(int n){//线性筛并计算莫比乌斯函数
	static bool flag[SIZE]={0};
	static int ptot=0,p[SIZE];
	for(int i=1;i<=n;i++) miu[i]=1;
	for(int i=2;i<=n;i++){
		if(!flag[i]) p[++ptot]=i,miu[i]=-1;
		for(int j=1;j<=ptot;j++){
			if((LL)i*p[j]>n) break;
			flag[i*p[j]]=true,miu[i*p[j]]=-miu[i];
			if(i%p[j]==0){
				miu[i*p[j]]=0;
				break;
			}
		}
	}
	for(int i=1;i<=n;i++) miu[i]=(miu[i-1]+(LL)i*i*miu[i])%MOD;
}
\end{lstlisting} 


\subsection{lucas}

\subsection{高斯消元}

\subsection{异或方程组高斯消元}
用高斯消元法解异或方程组,求秩
\begin{lstlisting}[language=C]
bool Gauss_Jordan(bool A[SIZE][SIZE],int n,int m){//是否有解
	//行数0~n-1,列数0~m(m这一列是常数)
	for(int i=0;i<n&&i<m;i++){
		int p=i;
		for(int k=i+1;k<n;k++) if(A[k][i]>A[p][i]) p=k;
		if(!A[p][i]) continue;
		for(int k=0;k<=m;k++) swap(A[i][k],A[p][k]);
		for(int j=0;j<n;j++){
			if(j==i||!A[j][i]) continue;
			for(int k=0;k<=m;k++) A[j][k]^=A[i][k];
		}
	}
	//两种0=1的姿势
	for(int i=0;i<n&&i<m;i++)
		if(A[i][i]==0&&A[i][m]!=0) return false;
	for(int i=m;i<n;i++)
		if(A[i][m]!=0) return false;
	return true;
}
int xor_rank(ll A[SIZEN][SIZEN],int b[SIZEN],int n,int m){
	//使用高斯约当消元法,求异或方程组的秩
	//A是异或方程组,b是方程右边的常数向量,n是行数,m是列数
	int i,j,k;
	for(i=1;i<=n;i++){//第i次消元
		for(k=i;k<=m;k++){//选取消元列
			for(j=i;j<=n;j++){//选取消元行
				if(A[j][k]) goto BREAK;
			}
		}
		BREAK:;
		if(j==n+1) break;//找不到可消元的元素了
		if(i!=j){//交换行
			for(int km=1;km<=m;km++) swap(A[i][km],A[j][km]);
			swap(b[i],b[j]);
		}
		if(i!=k) for(int km=1;km<=n;km++) swap(A[km][i],A[km][k]);//交换列
		for(j=1;j<=n;j++){//消元
			if(j==i) continue;
			if(A[j][i]){
				for(int km=1;km<=m;km++) A[j][km]^=A[i][km];
				b[j]^=b[i];
			}
		}
	}
	return i-1;
}
\end{lstlisting}


\subsection{BSGS}
使用大步小步算法计算离散对数
\begin{lstlisting}[language=C]
ll quickpow(ll a,ll n){//这个是来卖萌的
	ll ans=1;
	while(n){
		if(n&1) ans=(ans*a)%MOD;
		a=(a*a)%MOD;
		n>>=1;
	}
	return ans;
}
void extend_gcd(ll a,ll b,ll &x,ll &y,ll &d){
	if(b==0){d=a;x=1;y=0;}
	else{extend_gcd(b,a%b,y,x,d);y-=(a/b)*x;}
}
ll inverse(ll a){//a对模MOD的逆元,无解返回-1
	ll x,y,d;
	extend_gcd(a,MOD,x,y,d);
	return d==1?(x+MOD)%MOD:-1;//这个+不可少,因为x可能为负数
}
ll dclog(ll a,ll b){//求解方程:a^x=b(模 MOD),返回最小解,无解返回-1
	//采用大步小步法
	map<ll,ll> base;
	ll m=(ll)sqrt(MOD+0.5),e=1,i,v;
	v=inverse(quickpow(a,m));
	base[1]=0;
	for(i=1;i<m;i++){
		e=(e*a)%MOD;
		if(!base.count(e)) base[e]=i;
	}
	for(i=0;i<=MOD/m;i++){
		if(base.count(b)) return (i*m+base[b])%(MOD-1);
		b=(b*v)%MOD;
	}
	return -1;
}

\end{lstlisting}


\subsection{FFT}
\begin{lstlisting}[language=C]
const int SIZEN=800010,BASE=10;
const double PI=acos(-1.0),eps=1e-6;
class Complex{//复数
public:
	double x,y;//x+yi
	Complex(double x_=0,double y_=0){x=x_,y=y_;}
	void print(void){cout<<"("<<x<<","<<y<<")";}
	void clear(void){x=y=0;}
};
void print(Complex a){printf("%.8lf+%.8lfi",a.x,a.y);}
void swap(Complex &a,Complex &b){Complex c=a;a=b,b=c;}
Complex operator + (Complex a,Complex b){return Complex(a.x+b.x,a.y+b.y);}
Complex operator - (Complex a,Complex b){return Complex(a.x-b.x,a.y-b.y);}
Complex operator * (Complex a,Complex b){return Complex(a.x*b.x-a.y*b.y,a.x*b.y+b.x*a.y);}
Complex operator * (Complex a,double b){return Complex(a.x*b,a.y*b);}
Complex operator / (Complex a,double b){return Complex(a.x/b,a.y/b);}
class Poly{
public:
	int n;
	Complex s[SIZEN];
	void Initialize(char str[]){
		n=strlen(str);
		for(int i=0;i<n;i++) s[i]=Complex(str[n-1-i]-'0',0);
	}
	void Read(void){
		static char str[SIZEN];
		scanf("%s",str);
		Initialize(str);
	}
	void Assign(char str[]){
		static int a[SIZEN];
		int len;
		for(int i=0;i<n;i++) a[i]=int(s[i].x+0.5);
		for(len=0;len<n||a[len];len++){
			a[len+1]+=a[len]/BASE;
			a[len]%=BASE;
		}
		while(len>1&&!a[len-1]) len--;
		for(int i=0;i<len;i++) str[i]=a[len-1-i]+'0';
		str[len]=0;
	}
	void Print(void){
		static char str[SIZEN];
		Assign(str);
		printf("%s\n",str);
	}
	void Rader_Transform(void){
		int j=0,k;
		for(int i=0;i<n;i++){
			if(j>i) swap(s[i],s[j]);
			k=n;
			while(j&(k>>=1)) j&=~k;
			j|=k;
		}
	}
	void FFT(bool type){//type=1是求值(系数求点),type=0是插值(点求系数)
		Rader_Transform();//不能忘了
		double pi=type?PI:-PI;
		Complex w0;
		for(int step=1;step<n;step<<=1){
			double alpha=pi/step;
			Complex wn(cos(pi/step),sin(pi/step)),wk(1.0,0.0);
			for(int k=0;k<step;k++){
				for(int Ek=k;Ek<n;Ek+=step<<1){
					int Ok=Ek+step;
					Complex t=wk*s[Ok];
					s[Ok]=s[Ek]-t;
					s[Ek]=s[Ek]+t;
				}
				wk=wk*wn;
			}
		}
		if(!type) for(int i=0;i<n;i++) s[i]=s[i]/n;
	}
	void operator *= (Poly &b){
		int S=1;while(S<n+b.n) S<<=1;
		n=b.n=S;
		FFT(true);
		b.FFT(true);
		for(int i=0;i<n;i++) s[i]=s[i]*b.s[i];
		FFT(false);
	}
};
\end{lstlisting}


\subsection{素数测试}

\section{数据结构}

\subsection{splay}

\subsection{treap}

\subsection{主席树}

\subsection{可并堆}

\subsection{KDtree}

\subsection{可持久化线段树}

\subsection{可持久化trie}

\section{字符串}

\subsection{KMP}
\begin{lstlisting}[language=C]
void getnext(char *s,int n,int *next){//字符串s,文本下标0~n-1,结果存于f
	next[0]=-1;//这是一个特例
	int i=0,j=-1;
	while(i<n){
		while(j!=-1&&s[i]!=s[j]) j=next[j];
		next[++i]=++j;//如果在这个位置失配了应该去哪试图配
	}
}
int KMP(char *S,int n,char *T,int m){//目标串S的前n个中出现了几次模式串T的前m个
	int i=0,j=0;
	int ans=0;
	getnext(T,m,next);
	while(i<n){
		while(j!=-1&&S[i]!=T[j]) j=next[j];
		i++,j++;
		if(j==m) ans++;
	}
	return ans;
}
\end{lstlisting} 


\subsection{Manacher}
\begin{lstlisting}[language=C]
void Manacher(char S[],int N,int P[]){
	int mx=1,id=0;
	P[0]=1;
	for(int i=1;i<N;i++){
		if(mx>i) P[i]=min(P[2*id-i],mx-i);
		else P[i]=1;
		while(P[i]<=i&&S[i-P[i]]==S[i+P[i]]) P[i]++;
		if(i+P[i]>mx){
			mx=i+P[i];
			id=i;
		}
	}
}
\end{lstlisting}


\subsection{后缀数组}
后缀排序,输出SA数组以及height数组
\begin{lstlisting}[language=C] 
#include<cstdio>
#include<cstring>
#include<iostream>
#define ll long long 
using namespace std;
int n,p,q=1,k;
char ch[100005];
int sa[2][100005],rk[2][100005];
int a[100005],h[100005],v[100005];
void calsa(int *sa,int *rk,int *SA,int *RK)
{
	for(int i=1;i<=n;i++)v[rk[sa[i]]]=i;
	for(int i=n;i;i--)
		if(sa[i]>k)
			SA[v[rk[sa[i]-k]]--]=sa[i]-k;
	for(int i=n-k+1;i<=n;i++)
		SA[v[rk[i]]--]=i;
	for(int i=1;i<=n;i++)
		RK[SA[i]]=RK[SA[i-1]]+(rk[SA[i-1]]!=rk[SA[i]]||rk[SA[i-1]+k]!=rk[SA[i]+k]);
}
void getsa()
{
	for(int i=1;i<=n;i++)v[a[i]]++;
	for(int i=1;i<=30;i++)v[i]+=v[i-1];
	for(int i=1;i<=n;i++)
		sa[p][v[a[i]]--]=i;
	for(int i=1;i<=n;i++)
		rk[p][sa[p][i]]=rk[p][sa[p][i-1]]+(a[sa[p][i]]!=a[sa[p][i-1]]);
	for(k=1;k<n;k<<=1,swap(p,q))
		calsa(sa[p],rk[p],sa[q],rk[q]);
	for(int i=1,k=0;i<=n;i++)
	{
		int j=sa[p][rk[p][i]-1];
		while(a[i+k]==a[j+k])k++;
		h[rk[p][i]]=k;if(k)k--;
	}
}
int main()
{
	scanf("%s",ch+1);
	n=strlen(ch+1);
	for(int i=1;i<=n;i++)a[i]=ch[i]-'a'+1;
	getsa();
	for(int i=1;i<=n;i++)
		printf("%d ",sa[p][i]);
	cout<<endl;
	for(int i=2;i<=n;i++)
		printf("%d ",h[i]);
	return 0;
}
\end{lstlisting}

\subsection{AC自动机}
\begin{lstlisting}[language=C] 
#define Nil NULL
const int SIZES=1000010,SIZEN=100010;
class Node{
public:
	int flag;//是否敏感词
	Node* ch[26];
	Node* fail;
	Node(){
		flag=-1;
		memset(ch,0,sizeof(ch));
		fail=Nil;
	}
};
Node *root;
Node* step(Node *x,int k){
	while(true){
		if(x->ch[k]!=Nil) return x->ch[k];
		if(x==root) return root;
		x=x->fail;
	}
}
queue<Node*> Q;
void make_automaton(void){
	root->fail=root;
	Q.push(root);
	while(!Q.empty()){
		Node *x=Q.front();Q.pop();
		for(int i=0;i<26;i++){
			if(x->ch[i]!=Nil){
				if(x==root) x->ch[i]->fail=root;
				else x->ch[i]->fail=step(x->fail,i);
				Q.push(x->ch[i]);
			}
			else x->ch[i]=step(x,i);
		}
	}
}
\end{lstlisting} 


\subsection{后缀自动机(广义)}
\begin{lstlisting}[language=C]
typedef long long LL;
#define Nil NULL
const int SIZEN=100010;
const int SIZE_SAM=100000*2*20+10;
class Node{
public:
	int len;
	Node *suflink;
	Node *ch[10];
	void clear(void){
		len=0;
		suflink=Nil;
		memset(ch,0,sizeof(ch));
	}
	Node(void){clear();}
};
class Suffix_Automaton{
public:
	Node *root;
	int n;
	Node *lis[SIZE_SAM];
	int findpos(Node *a){
		if(a==Nil) return -1;
		for(int i=0;i<n;i++){
			if(lis[i]==a) return i;
		}
		return -2;
	}
	void print(Node *a){
		cout<<"地址: "<<findpos(a)<<endl;
		cout<<"len: "<<a->len<<endl;
		cout<<"suflink: "<<findpos(a->suflink)<<endl;
		for(int i=0;i<10;i++){if(a->ch[i]!=Nil){cout<<i<<" : "<<findpos(a->ch[i])<<endl;}}
		cout<<endl;
	}
	Node* new_Node(void){//新创立一个长度为l的节点
		Node *p=new Node;
		lis[n++]=p;
		return p;
	}
	void clear(void){
		n=0;
		root=new_Node();
	}
	Suffix_Automaton(void){clear();}
	LL calc(void){
		LL ans=0;
		for(int i=1;i<n;i++){
			Node *p=lis[i];
			ans+=p->len-p->suflink->len;
		}
		return ans;
	}
	Node* insert(Node *last,int k){
		//===和普通SAM的区别之处===
		Node *cur=last->ch[k];
		if(cur!=Nil&&cur->len==last->len+1) return cur;
		else{
			cur=new_Node();
			cur->len=last->len+1;
		}
		//=========================
		Node *p=last;
		while(p!=Nil&&p->ch[k]==Nil){
			p->ch[k]=cur;
			p=p->suflink;
		}
		if(p==Nil) cur->suflink=root;
		else{
			Node *q=p->ch[k];
			if(p->len+1==q->len) cur->suflink=q;
			else{
				Node *clone=new_Node();
				*clone=*q;
				clone->len=p->len+1;
				cur->suflink=clone;
				q->suflink=clone;
				while(p!=Nil&&p->ch[k]==q){
					p->ch[k]=clone;
					p=p->suflink;
				}
			}
		}
		return cur;
	}
}MT;
\end{lstlisting}


\subsection{回文自动机}

\section{图论}

\subsection{最短路}

\subsection{最小生成树}

%\subsection{网络流}
\subsection{网络流}
\subsubsection{Dinic}
\begin{lstlisting}[language=C] 
const int SIZEN=310,INF=0x7fffffff/2;
class EDGE{
public:
	int from,to,cap,flow;
};
vector<EDGE> edges;
vector<int> c[SIZEN];
bool visit[SIZEN]={0};//在需要找割集的时候就把这个放在外面
int S,T,N;//0~N
int depth[SIZEN]={0};
int cur[SIZEN]={0};
void addedge(int from,int to,int cap){
	EDGE temp;
	temp.from=from,temp.to=to,temp.cap=cap,temp.flow=0;
	edges.push_back(temp);
	temp.from=to,temp.to=from,temp.cap=0,temp.flow=0;
	edges.push_back(temp);
	int tot=edges.size()-2;
	c[from].push_back(tot);
	c[to].push_back(tot+1);
}
bool BFS(void){
	memset(visit,0,sizeof(visit));
	queue<int> Q;
	Q.push(S);visit[S]=true;depth[S]=0;
	while(!Q.empty()){
		int x=Q.front();Q.pop();
		for(int i=0;i<c[x].size();i++){
			EDGE &now=edges[c[x][i]];
			if(!visit[now.to]&&now.cap>now.flow){
				visit[now.to]=true;
				depth[now.to]=depth[x]+1;
				Q.push(now.to);
			}
		}
	}
	return visit[T];
}
int DFS(int x,int a){
	if(x==T||!a) return a;
	int flow=0,cf=0;
	for(int i=cur[x];i<c[x].size();i++){
		cur[x]=i;
		EDGE &now=edges[c[x][i]];
		if(depth[x]+1==depth[now.to]){
			cf=DFS(now.to,min(a,now.cap-now.flow));
			if(cf){
				flow+=cf;
				a-=cf;
				now.flow+=cf,edges[c[x][i]^1].flow-=cf;
			}
			if(!a) break;
		}
	}
	if(!flow) depth[x]=-1;
	return flow;
}
int Dinic(void){
	int flow=0;
	while(BFS()){
		memset(cur,0,sizeof(cur));
		flow+=DFS(S,INF);
	}
	return flow;
}
\end{lstlisting}


\subsection{tarjan}

\subsection{虚树}

\subsection{生成树计数}

\subsection{KM算法}

\subsection{lCT}

\subsection{2-SAT}

\subsection{带花树}

\subsection{曼哈顿最小生成树}

\section{计算几何}

\subsection{凸包}

\subsection{旋转卡壳}

\subsection{半平面交}

\subsection{辛普森积分}

\section{其它}

\subsection{高精度}

\maketitle 
\end{document}