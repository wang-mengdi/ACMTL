\documentclass[UTF8,a4paper]{ctexart}
\usepackage{color}
\usepackage{fancyhdr}
\usepackage{amsfonts}
\usepackage{graphicx}
\usepackage{listings}
\usepackage{lastpage}
\usepackage{multirow}
\usepackage{makecell}
\usepackage{setspace}
\usepackage{indentfirst}
\author{PKU cstdio,hzwer,Kuribohg}
\title{\Huge ACM模板}
\begin{document} 
\maketitle
\tableofcontents
\section{数学}

\subsection{1.线性筛}

\subsection{2.lucas}

\subsection{3.可并堆}

\subsection{4.高斯消元}

\subsection{5.BSGS}

\subsection{6.FFT}

\subsection{7.素数测试}

\section{数据结构}

\subsection{1.splay}

\subsection{2.treap}

\subsection{3.主席树}

\subsection{4.KDtree}

\subsection{5.可持久化线段树}

\subsection{6.可持久化trie}

\section{字符串}

\subsection{1.kmp}

\subsection{2.manachar}

\subsection{3.后缀数组}

\subsection{4.后缀自动机}

\subsection{5.回文自动机}

\section{图论}

\subsection{1.最短路}

\subsection{2.网络流}

\subsection{3.tarjan}

\subsection{4.虚树}

\subsection{5.生成树计数}

\subsection{6.KM算法}

\subsection{7.lCT}

\subsection{8.2-SAT}

\subsection{9.带花树}

\subsection{10.曼哈顿最小生成树}

\section{计算几何}

\subsection{1.凸包}

\subsection{2.旋转卡壳}

\subsection{3.半平面交}

\section{其它}

\subsection{1.高精度}

\maketitle 
\end{document}