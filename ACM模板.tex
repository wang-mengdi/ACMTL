\documentclass[UTF8,a4paper]{ctexart}
\usepackage{xcolor}
\usepackage{fancyhdr}
\usepackage{amsfonts}
%\usepackage{graphicx}
\usepackage{listings}
\usepackage{lastpage}
%\usepackage{multirow}
%\usepackage{makecell}
\usepackage{setspace}
\usepackage{indentfirst}

\usepackage{geometry}
\geometry{left=2.5cm,right=2.5cm,top=2.5cm,bottom=2.5cm}

\lstset{
  language=[ANSI]c,
  basicstyle=\small,
  numbers=left,
  keywordstyle=\color{blue},
  numberstyle={\tiny\color{lightgray}},
  stepnumber=1, %行号会逐行往上递增
  numbersep=5pt,
  showspaces=false,
  showtabs=false,
  tabsize=4,
  breaklines=tr,
  extendedchars=false %这一条命令可以解决代码跨页时,章节标题,页眉等汉字不显示的问题
}

\author{PKU cstdio,hzwer,Kuribohg}
\title{\Huge ACM模板}
\begin{document} 
\maketitle
\tableofcontents
\section{数学}

\input{./inc/LinearSieve.tex}

\subsection{lucas}

\subsection{高斯消元}

\input{./inc/Gauss_xor.tex}

\input{./inc/BSGS.tex}

\subsection{FFT}
\begin{lstlisting}[language=C]
const int SIZEN=800010,BASE=10;
const double PI=acos(-1.0),eps=1e-6;
class Complex{//复数
public:
	double x,y;//x+yi
	Complex(double x_=0,double y_=0){x=x_,y=y_;}
	void print(void){cout<<"("<<x<<","<<y<<")";}
	void clear(void){x=y=0;}
};
void print(Complex a){printf("%.8lf+%.8lfi",a.x,a.y);}
void swap(Complex &a,Complex &b){Complex c=a;a=b,b=c;}
Complex operator + (Complex a,Complex b){return Complex(a.x+b.x,a.y+b.y);}
Complex operator - (Complex a,Complex b){return Complex(a.x-b.x,a.y-b.y);}
Complex operator * (Complex a,Complex b){return Complex(a.x*b.x-a.y*b.y,a.x*b.y+b.x*a.y);}
Complex operator * (Complex a,double b){return Complex(a.x*b,a.y*b);}
Complex operator / (Complex a,double b){return Complex(a.x/b,a.y/b);}
class Poly{
public:
	int n;
	Complex s[SIZEN];
	void Initialize(char str[]){
		n=strlen(str);
		for(int i=0;i<n;i++) s[i]=Complex(str[n-1-i]-'0',0);
	}
	void Read(void){
		static char str[SIZEN];
		scanf("%s",str);
		Initialize(str);
	}
	void Assign(char str[]){
		static int a[SIZEN];
		int len;
		for(int i=0;i<n;i++) a[i]=int(s[i].x+0.5);
		for(len=0;len<n||a[len];len++){
			a[len+1]+=a[len]/BASE;
			a[len]%=BASE;
		}
		while(len>1&&!a[len-1]) len--;
		for(int i=0;i<len;i++) str[i]=a[len-1-i]+'0';
		str[len]=0;
	}
	void Print(void){
		static char str[SIZEN];
		Assign(str);
		printf("%s\n",str);
	}
	void Rader_Transform(void){
		int j=0,k;
		for(int i=0;i<n;i++){
			if(j>i) swap(s[i],s[j]);
			k=n;
			while(j&(k>>=1)) j&=~k;
			j|=k;
		}
	}
	void FFT(bool type){//type=1是求值(系数求点),type=0是插值(点求系数)
		Rader_Transform();//不能忘了
		double pi=type?PI:-PI;
		Complex w0;
		for(int step=1;step<n;step<<=1){
			double alpha=pi/step;
			Complex wn(cos(pi/step),sin(pi/step)),wk(1.0,0.0);
			for(int k=0;k<step;k++){
				for(int Ek=k;Ek<n;Ek+=step<<1){
					int Ok=Ek+step;
					Complex t=wk*s[Ok];
					s[Ok]=s[Ek]-t;
					s[Ek]=s[Ek]+t;
				}
				wk=wk*wn;
			}
		}
		if(!type) for(int i=0;i<n;i++) s[i]=s[i]/n;
	}
	void operator *= (Poly &b){
		int S=1;while(S<n+b.n) S<<=1;
		n=b.n=S;
		FFT(true);
		b.FFT(true);
		for(int i=0;i<n;i++) s[i]=s[i]*b.s[i];
		FFT(false);
	}
};
\end{lstlisting}


\subsection{素数测试}

\section{数据结构}

\subsection{splay}

\subsection{treap}

\subsection{主席树}

\subsection{可并堆}

\subsection{KDtree}

\subsection{可持久化线段树}

\subsection{可持久化trie}

\section{字符串}

\input{./inc/KMP.tex}

\input{./inc/Manacher.tex}

\input{./inc/SuffixArray.tex}

\subsection{AC自动机}
\begin{lstlisting}[language=C] 
#define Nil NULL
const int SIZES=1000010,SIZEN=100010;
class Node{
public:
	int flag;//是否敏感词
	Node* ch[26];
	Node* fail;
	Node(){
		flag=-1;
		memset(ch,0,sizeof(ch));
		fail=Nil;
	}
};
Node *root;
Node* step(Node *x,int k){
	while(true){
		if(x->ch[k]!=Nil) return x->ch[k];
		if(x==root) return root;
		x=x->fail;
	}
}
queue<Node*> Q;
void make_automaton(void){
	root->fail=root;
	Q.push(root);
	while(!Q.empty()){
		Node *x=Q.front();Q.pop();
		for(int i=0;i<26;i++){
			if(x->ch[i]!=Nil){
				if(x==root) x->ch[i]->fail=root;
				else x->ch[i]->fail=step(x->fail,i);
				Q.push(x->ch[i]);
			}
			else x->ch[i]=step(x,i);
		}
	}
}
\end{lstlisting} 


\input{./inc/Suffix_Automaton.tex}

\subsection{回文自动机}

\section{图论}

\subsection{最短路}

\subsection{最小生成树}

\subsection{斯坦纳树}
\begin{lstlisting}[language=C]
const int INF=0x7fffffff/2;
const int SIZEN=60;
int N,M,K;
vector<pair<int,int> > c[SIZEN];
void SPFA(int S,int dis[]){
	static bool inq[SIZEN];
	static queue<int> Q;
	for(int i=0;i<=N;i++) inq[i]=false,dis[i]=INF;
	while(!Q.empty()) Q.pop();
	dis[S]=0;Q.push(S);inq[S]=true;
	while(!Q.empty()){
		int x=Q.front();Q.pop();inq[x]=false;
		for(int i=0;i<c[x].size();i++){
			int u=c[x][i].first,w=c[x][i].second;
			if(dis[x]+w<dis[u]){
				dis[u]=dis[x]+w;
				if(!inq[u]){
					inq[u]=true;
					Q.push(u);
				}
			}
		}
	}
}
int F[SIZEN][1<<10]={0};//i为根,连通状态至少为s
int dis[SIZEN]={0};
void Steiner(void){
	for(int i=0;i<=N;i++){
		for(int j=0;j<(1<<(2*K));j++) F[i][j]=INF;
	}
	for(int i=1;i<=K;i++){
		F[i][1<<(i-1)]=0;
		F[N+1-i][1<<(i+K-1)]=0;
	}
	for(int s=0;s<(1<<(2*K));s++){
		for(int i=1;i<=N;i++){
			for(int t=s;t;t=(t-1)&s){
				F[i][s]=min(F[i][s],F[i][t]+F[i][s-t]);
			}
		}
		for(int i=1;i<=N;i++) c[0][i-1].second=F[i][s];
		SPFA(0,dis);
		for(int i=1;i<=N;i++) F[i][s]=min(F[i][s],dis[i]);
	}
}
\end{lstlisting}


%\subsection{网络流}
\subsection{网络流}
\subsubsection{Dinic}
\begin{lstlisting}[language=C] 
const int SIZEN=310,INF=0x7fffffff/2;
class EDGE{
public:
	int from,to,cap,flow;
};
vector<EDGE> edges;
vector<int> c[SIZEN];
bool visit[SIZEN]={0};//在需要找割集的时候就把这个放在外面
int S,T,N;//0~N
int depth[SIZEN]={0};
int cur[SIZEN]={0};
void addedge(int from,int to,int cap){
	EDGE temp;
	temp.from=from,temp.to=to,temp.cap=cap,temp.flow=0;
	edges.push_back(temp);
	temp.from=to,temp.to=from,temp.cap=0,temp.flow=0;
	edges.push_back(temp);
	int tot=edges.size()-2;
	c[from].push_back(tot);
	c[to].push_back(tot+1);
}
bool BFS(void){
	memset(visit,0,sizeof(visit));
	queue<int> Q;
	Q.push(S);visit[S]=true;depth[S]=0;
	while(!Q.empty()){
		int x=Q.front();Q.pop();
		for(int i=0;i<c[x].size();i++){
			EDGE &now=edges[c[x][i]];
			if(!visit[now.to]&&now.cap>now.flow){
				visit[now.to]=true;
				depth[now.to]=depth[x]+1;
				Q.push(now.to);
			}
		}
	}
	return visit[T];
}
int DFS(int x,int a){
	if(x==T||!a) return a;
	int flow=0,cf=0;
	for(int i=cur[x];i<c[x].size();i++){
		cur[x]=i;
		EDGE &now=edges[c[x][i]];
		if(depth[x]+1==depth[now.to]){
			cf=DFS(now.to,min(a,now.cap-now.flow));
			if(cf){
				flow+=cf;
				a-=cf;
				now.flow+=cf,edges[c[x][i]^1].flow-=cf;
			}
			if(!a) break;
		}
	}
	if(!flow) depth[x]=-1;
	return flow;
}
int Dinic(void){
	int flow=0;
	while(BFS()){
		memset(cur,0,sizeof(cur));
		flow+=DFS(S,INF);
	}
	return flow;
}
\end{lstlisting}


\subsection{tarjan}

\subsection{虚树}

\subsection{生成树计数}

\subsection{KM算法}

\subsection{lCT}

\subsection{2-SAT}

\subsection{带花树}

\subsection{曼哈顿最小生成树}

\section{计算几何}

\subsection{凸包}

\subsection{旋转卡壳}

\subsection{半平面交}

\subsection{辛普森积分}

\section{其它}

\subsection{高精度}

\maketitle 
\end{document}